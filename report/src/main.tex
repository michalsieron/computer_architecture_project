\documentclass{article}

% \usepackage{showframe}
% \usepackage[export]{adjustbox}
\usepackage[a4paper, vmargin=1in]{geometry}
\usepackage{graphicx}
\usepackage{polski}
\usepackage{indentfirst}
\usepackage{fancyhdr}
\usepackage{float}
\usepackage{subcaption}
\usepackage{multirow}
\usepackage{hyperref}

\pagestyle{fancy}
\fancyhf{}

\lhead{Architektura Komputerów 2}
\rhead{\thepage}

\begin{document}

% strona tytułowa
\begin{titlepage}
	\clearpage
	\thispagestyle{empty}
	\pagenumbering{gobble}
    \centering
    
	{\LARGE Architektura Komputerów 2 \par}
	
	\vspace{1.5cm}
	
	{\huge\bfseries Liczby zdenormalizowane\par}
	
	\vspace{1.5cm}
	{\large czwartek nieparzysty, 18:55\par}

	\vspace{1cm}
    \parbox{0.4\linewidth}{
	    \centering
	    {\Large Michał \textsc{Sieroń}\par}
	    {\large 256 259\par}
	}
    \hfill
    \parbox{0.4\linewidth}{
	    \centering
	    {\Large Paweł \textsc{Różański}\par}
	    {\large 252 772\par}
	}

	\vspace{1.5cm}
	{\large prowadzący\par}
	{\large dr inż. Piotr \textsc{Patronik}\par}
    
	\vfill

	{\large Informatyka Techniczna, Wydział Elektroniki\par}
	\vspace{0.5cm}
	{\large \today\par}
	\clearpage
	\pagenumbering{arabic}
\end{titlepage}

\tableofcontents
\newpage

\section{Wstęp}
Zadanie projektowe polegało na analizie zawartości artykułu i implementacji przedstawionych układów w języku Verilog.
Z powodu ograniczonego czasu na wykonanie projektu możliwe było zaimplemenetowanie jedynie układów A1 oraz M1 przedstawionych w artykule \cite{art:old}.

Następnie zaimplementowano odpowiadajace im układy zgodne ze standardem IEEE 754 \cite{art:ieee754}.
Dokonano również porównania błędów wynikajacych z użycia danej reprezentacji liczby zmiennoprzecinkowej. 


\section{Opis koncepcji zapisu i formatu}
% artykuł po polsku

\section{Opis metody}
% python wywołujacy veriloga

\newpage
\section{Opis implementacji}
% Opis kodu w verilogu + schematy A1, M1

\newpage
\section{Różnice między metodami}
% Opis i schematy IEEE 754 porównanie

\section{Narzędzia}
% Użyte narzędzia

\section{Opis sposobu}
% (w jaki sposób zrobilismy to badanie) - w jaki sposób były liczone błędy, ilość kropeczek, czego użyliśmy do uzyskania wyników

\newpage
\section{Wyniki pomiarów}
% Wykresy 

\newpage
\section{Wnioski}


\newpage
\phantomsection
\addcontentsline{toc}{section}{Literatura}
\bibliography{bibliography}
\bibliographystyle{plabbrv}


\end{document}
