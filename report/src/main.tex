\documentclass{article}

% \usepackage{showframe}
% \usepackage[export]{adjustbox}
% \usepackage[a4paper, vmargin=1in]{geometry}
\usepackage{graphicx}
\usepackage{polski}
\usepackage{indentfirst}
\usepackage{fancyhdr}
\usepackage{float}
\usepackage{subcaption}
\usepackage{multirow}
\usepackage{hyperref}

\pagestyle{fancy}
\fancyhf{}

\lhead{Architektura Komputerów 2}
\rhead{\thepage}

\begin{document}

% strona tytułowa
\begin{titlepage}
	\clearpage
	\thispagestyle{empty}
	\pagenumbering{gobble}
    \centering
    
	{\LARGE Architektura Komputerów 2 \par}
	
	\vspace{1.5cm}
	
	{\huge\bfseries Liczby zdenormalizowane\par}
	
	\vspace{1.5cm}
	{\large czwartek nieparzysty, 18:55\par}

	\vspace{1cm}
    \parbox{0.4\linewidth}{
	    \centering
	    {\Large Michał \textsc{Sieroń}\par}
	    {\large 256 259\par}
	}
    \hfill
    \parbox{0.4\linewidth}{
	    \centering
	    {\Large Paweł \textsc{Różański}\par}
	    {\large 252 772\par}
	}

	\vspace{1.5cm}
	{\large prowadzący\par}
	{\large dr inż. Piotr \textsc{Patronik}\par}
    
	\vfill

	{\large Informatyka Techniczna, Wydział Elektroniki\par}
	\vspace{0.5cm}
	{\large \today\par}
	\clearpage
	\pagenumbering{arabic}
\end{titlepage}

\tableofcontents
\newpage

\section{Wstęp}
Zadanie projektowe polegało na analizie zawartości artykułu i implementacji przedstawionych układów w języku \emph{Verilog}.
Z powodu ograniczonego czasu na wykonanie projektu możliwe było zaimplemenetowanie jedynie układów \emph{A1} oraz \emph{M} przedstawionych w artykule \cite{art:old}.

Następnie zaimplementowano odpowiadajace im układy zgodne ze standardem \emph{IEEE-754} \cite{art:ieee754}.
Dokonano również porównania błędów wynikajacych z użycia danej reprezentacji liczby zmiennoprzecinkowej. 


\section{Opis koncepcji zapisu i formatu}
Koncepcja formatu liczb zmiennoporzecinkowych, zaproponowanego w artykule \cite{art:old} wzięła się z~obserwacji, że standard \emph{IEEE-754} nie został stworzony z myślą o systemach wbudowanych.
Eliminacja logiki normalizującej powinna obniżyć koszt produkcji układu, a wpływ na precyzję obliczeń nie powinien mieć znaczenia w docelowych zastosowaniach.
Normalizacja liczby jest sktukiem używania ukrytej jedynki w liczbach znormalizowanych.
Sprawia to, że pojawia się wyjątek, który trzeba obsłużyć w sprzęcie.
Proponowany format pozbywa się ukrytej jedynki, kosztem jednego z bitów mantysy.
Konsekwencją tego jest zmniejszona precyzja liczb.


\section{Opis metody}
Wszystkie układy zostały zaimplementowane w języku opisu sprzętu \emph{Verilog}.
Jednak druga część projektu, która je ze sobą łączy i porównuje z wartością referencyjną, została napisana w języku \emph{Python}.
Dla zadanej ilości przypadków testowych generowaliśmy tyle samo par liczb typu \texttt{float}.
W języku \emph{Python}, typ \texttt{float} odpowiada liczbie zmiennoprzecinkowej o podwójnej precyzji.
Wobec tego konieczne było przekonwertowanie wygenerowanych liczb na liczbę zmiennoprzecinkową o pojedynczej precyzji.
W tym celu napisaliśmy funkcję \texttt{py2float} w języku \emph{C}, która zamienia wartość typu \texttt{double} na \texttt{float}.
Tak otrzymane wartości były następnie zamieniane na ich szesnastkową reprezentację.
W tym celu musieliśmy uprzednio otrzymać bajtową reprezentację danej liczby, która z~kolei była zamieniana na liczbę typu \texttt{int}, z której w końcu mogliśmy otrzymać reprezentację w systemie szesnastkowym.

Konwersję z liczby znormalizowanej na zdenormalizowaną zaimplementowaliśmy w tym samym skrypcie.
W ten sam sposób co wcześniej, otrzymywaliśmy zapis szesnastkowy liczby lecz denormalizacja liczby wymaga operacji bitowych.
Przez fakt użycia \emph{Pythona} konieczne było do tego zamienienie liczby na listę bitów (w tym przypadku liczb całkowitych typu \texttt{int} o wartościach 0 lub 1).
Tak otrzymana lista była następnie odpowiednio dzielona na części znaku, wykładnika i mantysy.
Mantysa była przesuwana o jedną pozycję w prawo, a wykładnik zwiększany o jeden.
Tak zmodyfikowaną reprezentację bitową zamienialiśmy z~powrotem na zapis szesnastkowy.

Wykorzystując wygenerowane pary liczb tworzyliśmy pliki \emph{Veriloga} wykorzystujące napisane przez nas moduły opisane w sekcji \ref{sec:implementacja}.
Utworzone pliki były następnie uruchamiane, a wyniki zapisywane w pliku \emph{csv} do dalszego przetwarzania.


\section{Opis implementacji}\label{sec:implementacja}

\subsection{Układy dodawania}
Implementacja układu dodawania liczb zmiennoprzecinkowych została wykonana na podstawie opisu oraz 
modelu układu z artykułu \cite{art:old}.
Według nazewnictwa z artykułu układ dodawania, który odwzorowaliśmy, to \emph{A1}. 
Jest to najprostsza wersja dodawania dwóch liczb zdenormalizowanych.

Implementację układu zaczęliśmy od dokładnego przeczytania opisu przedstawionego w~artykule, a następnie stworzeniu podstawowych bloków do obliczeń zawartych na diagramie \ref{fig:diagram_add_denorm}.

% schemat dodawania zdenormalizowanego
\begin{figure}[H]
	\centering
	\includegraphics[width=\textwidth]{figures/diagram_add_denorm.pdf}
	\caption{Schemat układu dodawania - zdenormalizowane}
	\label{fig:diagram_add_denorm}
\end{figure}

Na początku układ oblicza różnicę wykładników, żeby następnie wyrównać drugą z nich do tej samej wartości wykładnika.
Liczba z mniejszym wykładnikiem jest przesuwana w prawo o ilość bitów równą wartości różnicy wykładników.
Jeżeli liczby mają różne znaki, to wyrównywana liczba jest dodatkowo negowana.
Umożliwia to sumowanie mantys w U2 wykorzystując obliczony wcześniej bit \texttt{different\_signs} jako przeniesienie wejściowe.
Wewnątrz sumatora wykrywane jest przepełnienie i po obliczeniu wartości bezwzględnej wyniku, wykorzystywane do ewentualnego przesunięcia go w prawo o jeden bit.
W takim przypadku zwiększany jest też wykładnik wyjściowy.
Wyznaczenie znaku w przypadku liczb zdenormalizowanych nie jest łatwe.
Wynika to z faktu, że nawet gdy wykładnik jednej z liczb jest większy od drugiej, to różnica w mantysach może być jeszcze większa.
Wobec tego, konieczne jest wykorzystanie znaków różnicy wykładników, sumy mantys oraz bitu \texttt{ovf} informującym o przepełnieniu.

Zaimplementowany przez nas układ dodawania liczb znormalizowanych oparty jest o wersję zdenormalizowaną.
Różnice zaznaczyliśmy na rysunku \ref{fig:diagram_add_ieee754} na zielono.

% schemat dodawania IEEE-754
\begin{figure}[H]
	\centering
	\includegraphics[width=\textwidth]{figures/diagram_add_ieee754.pdf}
	\caption{Schemat układu dodawania - \emph{IEEE-754}}
	\label{fig:diagram_add_ieee754}
\end{figure}

% napisać o znaku
Pierwszą zmianą jest warunkowe dodanie ukrytej jedynki.
Drugą zaznaczoną zmianą jest blok normalizacyjny.
Mantysa jest przesuwana w lewo dopóki na pozycji ukrytego bitu nie pojawi się 1.
Natomiast inną zmianą, która nie jest uwzględniona na schemacie, jest szerokość bitowa sumy mantys.
W wersji liczb znormalizowanych ma ona 48 bitów.

\subsection{Układy mnożenia}

Kolejnym układem, który zaimplementowaliśmy, był układ mnożący, który podobnie jak układ dodawania pochodzi z artykułu \cite{art:old}.
Autorzy nadali mu nazwę \emph{M}.
Jest to pierwsza wersja zaproponowanej przez autorów implementacji układu mnożenia liczb zdenormalizowanych.

% schemat mnożenia zdenormalizowanego
\begin{figure}[H]
	\centering
	\includegraphics[width=\textwidth]{figures/diagram_mul_denorm.pdf}
	\caption{Schemat układu mnożenia - zdenormalizowane}
	\label{fig:diagram_mul_denorm}
\end{figure}

W przeciwieństwie do sumatora, wyznaczenie znaku jest prostą operacją \texttt{XOR}.
Wykładnik jest sumą wykładników wejściowych z odjętym obciążeniem.
Autorzy przyjęli założenie, że operacja mnożenia mantys, zawsze zakończy się przepełnieniem.
Wobec tego wykładnik wyjściowy jest zawsze zwiększany o jeden, a mantysa przesuwana w prawo.
W przypadku braku wystąpienia przepełnienia, tracony jest jeden bit precyzji.

Zmiany względem układu mnożącego liczby zgodne ze standardem \emph{IEEE-754} są w tym przypadku niewielkie.

% schemat mnozenia IEEE-754
\begin{figure}[H]
	\centering
	\includegraphics[width=\textwidth]{figures/diagram_mul_ieee754.pdf}
	\caption{Schemat układu mnożenia - \emph{IEEE-754}}
	\label{fig:diagram_mul_ieee754}
\end{figure}

Ponieważ wynik mnożenia dwóch licz znormalizowanych gwarantuje pojawienie się jedynki na jednym z dwóch najstarszych bitów wyniku, normalizacja sprowadza się do warunkowego przesunięcia w prawo w zależności od wystąpienia przepełnienia.


\section{Narzędzia}
Narzędziami, które zostały przez nas użyte były między innymi 3 języki programowania.
Najbardziej korzystaliśmy z języka \emph{Verilog}, który posłużył nam do implementacji układów dodawania i mnożenia.
W \emph{C} napisaliśmy funkcje konwertujące wartości typu \texttt{double} na \texttt{float} konieczne do porówania wyników.
\emph{Python} służył jako język, z którego wywoływane były wszystkie polecenia kompilujące wcześniej wspomniane funkcje w \emph{C}, generujące oraz uruchamiające pliki testowe \emph{Veriloga}.
Przy użyciu \emph{Pythona} obliczaliśmy również błędy obliczeniowe wynikające z formatu zdenormalizowanych liczb zmiennoprzecinkowych oraz wykresy je prezentujące.
Do tworzenia wykresów posłużyliśmy się biblioteką \emph{Matplotlib}.
Narzędzie \emph{Icarus Verilog} posłużyło nam do kompilacji i~uruchamiania modułów napisanych w języku \emph{Verilog}.
Natomiast do sprawdzenia poprawności wyników poszczególnych bloków i debugowania programu używaliśmy programu \emph{GTKWave}.
Do utworzenia diagramów została użyta aplikacja \url{diagrams.net}.
Całość kodu była tworzona w programie \emph{Visual Studio Code}.
Testowanie i uruchamianie miało miejsce w systemie Ubuntu na maszynie wirtualnej \emph{WSL 2}.


\section{Opis sposobu testowania}
Stworzone układy sumatora i mnożenia liczb zdenormalizowanych zostały przetestowane pod względem różnic pomiędzy wynikami w swoich odpowiednikach w implementacji \emph{IEEE-754}.
Testowanie rozpoczeliśmy od wygenerowania liczb pseudolosowych w zakresie, w którym zadbaliśmy o to, aby nie doszło do przepełnienia wykładnika.
Wygenerowana liczba utworzona została w języku \emph{Python}, więc była podwójnej precyzji.
Przy pomocy zaimplementowanej przez nas funkcji, zmieniliśmy ją na liczbę pojedynczej precyzji.

Kolejnym krokiem w testowaniu było wykonanie odpowiednich dodawań lub mnożeń w zależności od testowanych układów.
Program następnie startował symulator uruchamiający kolejne układy.
Pierwszym z nich był zaimplementowany przez nas układ działający na liczbach zdenormalizowanych.
Drugim był układ częściowo zgodny ze standardem \emph{IEEE-754}.
Do tego obliczaliśmy wynik operacji na liczbach podwójnej precyzji, którego używaliśmy jako wartości referencyjnej do porównywania wyników.
Tak otrzymane liczby były zapisywane do plików \emph{CSV}.

Testowanie przeprowadzilismy dla następujących ilości wyników 100, 1 000, 10 000, 100 000, 1 000 000, 10 000 000.
Uznaliśmy, że wykresy najlepiej obrazowały otrzymane wyniki przy 1 000 000 punktów.
Możliwe jest wtedy zaobserwowanie szczegółów pozwalających na analizę wyników.
Przy większych ilościach punktów, wykresy stawały się nieczytelne.

Następnym krokiem było porównanie poszczególnych wartości liczb zdenormalizowanych i odpowiadającym im liczb referencyjnych.
Dla każdego wyniku obliczyliśmy błąd względny używając wyniku liczb podwójnej precyzji jako wartości referencyjnej rysunki \ref{fig:add_relative}, \ref{fig:sub_relative}, \ref{fig:mul_relative}.
Dodatkowo utworzyliśmy wykresy prezentujące błąd bezwzględny w ULP - rysunki \ref{fig:add_ulp}, \ref{fig:sub_ulp}, \ref{fig:mul_ulp}.


\section{Wyniki pomiarów}
\begin{figure}[H]
	\centering
	\includegraphics[height=0.4\textheight]{figures/add_relative.png}
	\caption{Błąd względny - dodawanie}
	\label{fig:add_relative}
\end{figure}

\begin{figure}[H]
	\centering
	\includegraphics[height=0.4\textheight]{figures/add_ulp.png}
	\caption{Błąd bezwzględny - dodawanie}
	\label{fig:add_ulp}
\end{figure}


\begin{figure}[H]
	\centering
	\includegraphics[height=0.4\textheight]{figures/sub_relative.png}
	\caption{Błąd względny - odejmowanie}
	\label{fig:sub_relative}
\end{figure}

\begin{figure}[H]
	\centering
	\includegraphics[height=0.4\textheight]{figures/sub_ulp.png}
	\caption{Błąd bezwzględny - odejmowanie}
	\label{fig:sub_ulp}
\end{figure}


\begin{figure}[H]
	\centering
	\includegraphics[height=0.4\textheight]{figures/mul_relative.png}
	\caption{Błąd względny - mnożenie}
	\label{fig:mul_relative}
\end{figure}

\begin{figure}[H]
	\centering
	\includegraphics[height=0.4\textheight]{figures/mul_ulp.png}
	\caption{Błąd bezwzględny - mnożenie}
	\label{fig:mul_ulp}
\end{figure}

\section{Wnioski}


\newpage
\phantomsection
\addcontentsline{toc}{section}{Literatura}
\bibliography{bibliography}
\bibliographystyle{plabbrv}


\end{document}
